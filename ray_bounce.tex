\documentclass{article}

\usepackage[a4paper, total={6in, 8in}]{geometry}
\usepackage{amsmath}
\usepackage{amsfonts}
\usepackage{amsthm}
\usepackage{indentfirst}
\usepackage{hyperref}

\usepackage{algorithm}
\usepackage{algpseudocode}

\newcommand{\norm}[1]{\left\lVert #1 \right\rVert}
\newcommand{\abs}[1]{\left\lvert #1 \right\rvert}
\newcommand{\Div}[0]{\nabla\cdot}
\newcommand{\Curl}[0]{\nabla\times}

\newtheorem{theorem}{Theorem}[section]
\newtheorem{corollary}{Corollary}[theorem]
\newtheorem{lemma}[theorem]{Lemma}

\theoremstyle{plain}
\newtheorem*{remark}{Remark}

\title{Scattered field from plane wave on infinite flat plate}
\author{Ian Holloway}
\date{September 2024}

\begin{document}
\maketitle

\section{Introduction}\label{sec_intro}

The scattered magnetic field is
\begin{equation}
	\mathbf{H}(x) = \int \mathbf{J}(x') \times \nabla_{x'} G(x,x') dS'
\end{equation}
\begin{equation}
	\mathbf{H}(x)
	= \int \mathbf{J}(x') \times \nabla_{x'} \frac{e^{-ik\norm{x-x'}}}{4\pi\norm{x-x'}} dS'
\end{equation}
\begin{equation}
	\mathbf{H}(x)
	= -\int \mathbf{J}(x') \times \hat{R} \left( ik + \frac{1}{R} \right) \frac{e^{-ik\norm{x-x'}}}{4\pi\norm{x-x'}} dS'
\end{equation}
\begin{equation}
	\mathbf{H}(x)
	= -\int \mathbf{J}(x') \times \frac{x-x'}{\norm{x-x'}}
	\left( ik + \frac{1}{\norm{x-x'}} \right)
	\frac{e^{-ik\norm{x-x'}}}{4\pi\norm{x-x'}} dS'
\end{equation}
Assume that the surface is flat and is WLOG aligned with the $x_1x_2$ plane.
\begin{equation}
	\mathbf{H}(x)
	= -\int_{\mathbb{R}^2} \mathbf{J}(x') \times \frac{x-x'}{\norm{x-x'}}
	\left( ik + \frac{1}{\norm{x-x'}} \right)
	\frac{e^{-ik\norm{x-x'}}}{4\pi\norm{x-x'}} dx_1'dx_2'
\end{equation}
Let the incident field be a plane wave
\begin{equation}
	\mathbf{H}_i = \mathbf{H}e^{-ik \hat{p} \cdot x},
\end{equation}
which induces a plane wave current
\begin{equation}
	\mathbf{J}(x') = \mathbf{J}e^{-ik(ax_1' + bx_2')}
\end{equation}
with
\begin{equation}
	\mathbf{J} = 2\hat{n}\times\mathbf{H},
	\quad \begin{bmatrix} a & b & 0 \end{bmatrix} = \hat{p} - \left( \hat{n}\cdot\hat{p} \right)\hat{n}.
\end{equation}
Now,
\begin{equation}
	\mathbf{H}(x)
	= -\mathbf{J} \times \int_{\mathbb{R}^2}
	\frac{x-x'}{\norm{x-x'}}
	\left( ik + \frac{1}{\norm{x-x'}} \right)
	\frac{1}{4\pi\norm{x-x'}}
	e^{-ik\left( \norm{x-x'} + ax_1' + bx_2' \right)} dx_1'dx_2'
\end{equation}


The stationary phase approximation in two dimensions is
\begin{equation}
	\int g(x) e^{ikf(x)}
	\approx \frac{2\pi}{k} g(x^*)
	\abs{\det \nabla^2 f(x^*)}^{-1/2}
	e^{ikf(x^*) + \frac{i\pi}{4}\text{sign}\nabla^2 f(x^*)}
\end{equation}
for
\begin{equation}
	\nabla f(x^*) = 0.
\end{equation}
To apply this approximation to the radiation intergral,
we set
\begin{equation}
	g(x') = 
	\frac{x-x'}{\norm{x-x'}}
	\left( ik + \frac{1}{\norm{x-x'}} \right)
	\frac{1}{4\pi\norm{x-x'}}
\end{equation}
and
\begin{equation}
	f(x') = -\norm{x-x'} - ax_1' - bx_2'.
\end{equation}

To find the stationary point of the radiation intergral, first WLOG let $x=(0,0,h)$,
so the gradient of the phase function is
\begin{equation}
	\nabla_{x'} f(x') = 
	\begin{bmatrix}
		\frac{x_1 - x_1'}{\norm{x-x'}} - a \\
		\frac{x_2 - x_2'}{\norm{x-x'}} - b
	\end{bmatrix}
	=
	\begin{bmatrix}
		\frac{-x_1'}{\sqrt{x_1^{'2} + x_2^{'2} + h^2}} - a \\
		\frac{-x_2'}{\sqrt{x_1^{'2} + x_2^{'2} + h^2}} - b
	\end{bmatrix}
\end{equation} % note <x_1',x_2'> = c * <a,b>, then solve for c
We compute
\begin{equation}
	x^{'*} = \left( \frac{-ha}{\sqrt{ 1 - a^2 - b^2 }}, \frac{-hb}{\sqrt{ 1 - a^2 - b^2 }}, 0 \right)
\end{equation}
so
\begin{equation}
	R^* = \frac{h}{\sqrt{1-a^2-b^2}},
	\quad \hat{R}^* - \left( \hat{n}\cdot\hat{R}^* \right)\hat{n} = \hat{p} - \left( \hat{n}\cdot\hat{p} \right)\hat{n}
	\quad\Rightarrow\quad \hat{R}^* =  \hat{p}\text{, } \hat{p} - 2\left( \hat{n}\cdot\hat{p} \right)\hat{n},
\end{equation}
which is consistent with Snell's law.


The Hessian of the phase function at the stationary point is
\begin{equation}
	\nabla^2_{x'} f(x^{'*}) =
	\frac{1}{R}\left( \hat{R}\hat{R}^T - I \right) =
	\frac{\sqrt{1-a^2-b^2}}{h}
	\begin{bmatrix}
		a^2-1 & ab \\
		ab & b^2-1
	\end{bmatrix}
\end{equation}
with determinant
\begin{equation}
	\det \nabla^2 f(x^{'*})
	= \frac{1-a^2-b^2}{R^{*2}}
	= \frac{\left(\hat{n}\cdot\hat{R}^*\right)^2}{R^{*2}}
\end{equation}
and eigenvalues
\begin{equation}
	\lambda\left( \nabla^2 f(x^{'*}) \right) = 
	-1, -\left( 1-a^2-b^2 \right)
	\quad\Rightarrow\quad \text{sign}\nabla^2 f(x^{'*}) = -2.
\end{equation}


The approximated radiation integral is
\begin{equation}
	\mathbf{H}(x) \approx
	-\frac{2\pi}{k}
	\mathbf{J} \times \hat{R}^*
	\left( ik + \frac{1}{R^*} \right)
	\frac{1}{4\pi R^*}
	\left( \frac{\left(\hat{n}\cdot\hat{R}^*\right)^2}{R^{*2}} \right)^{-1/2}
	e^{-ikR^*}
	e^{-ik(ax_1^{'*} + bx_2^{'*})}
	e^{\frac{-i\pi}{2}}
\end{equation}
\begin{equation}
	=
	-\frac{1}{2\left(\hat{n}\cdot\hat{R}^*\right)}
	\left( 1 + \frac{1}{ikR^*} \right)
	\mathbf{J}(x^{'*}) \times \hat{R}^*
	e^{-ik\hat{R}^*\cdot (R^*\hat{R}^*)}
\end{equation}
\begin{equation}
	\approx
	-\frac{1}{2\left(\hat{n}\cdot\hat{R}^*\right)}
	\mathbf{J}(x^{'*}) \times \hat{R}^*
	e^{-ik\hat{R}^*\cdot (R^*\hat{R}^*)}
\end{equation}
\begin{equation}
	=
	-\frac{1}{\left(\hat{n}\cdot\hat{R}^*\right)}
	\left( \hat{n} \times \mathbf{H}_i(x^{'*}) \right) \times \hat{R}^*
	e^{-ik\hat{R}^*\cdot (R^*\hat{R}^*)}
\end{equation}
Note that this is a plane wave.
It travels in a single direction, $\hat{R}$, and
there is no leading factor of $1/R$.

When we set the scattered direction to the incident direction, $\hat{R}^*=\hat{p}$,
and keep in mind that the amplitude of the incident wave is perpendicular to its propagation direction,
the scattered amplitude is
\begin{equation}
	-\frac{1}{\left(\hat{n}\cdot\hat{p}\right)}
	\left( \left(\hat{n}\cdot\hat{p}\right)\mathbf{H} - \left(\mathbf{H}\cdot\hat{p}\right)\hat{n} \right)
	=-\mathbf{H},
\end{equation}
which perfectly cancels the incident wave.


\section{Derivation of PO from MFIE}

We start with the MFIE problem defined by
\begin{equation}
	2\hat{n}\times\mathbf{H}_i(x)
	= \mathbf{J}(x)
	- 2\hat{n}\times\int \mathbf{J}(x') \times \nabla_{x'} G(x,x') dS'.
\end{equation}
We then break the domain of integration into pieces
\begin{equation}
	2\hat{n}\times\mathbf{H}_i(x)
	= \mathbf{J}(x)
	- \sum_n 2\hat{n}\times\int_{\Omega_n} \mathbf{J}(x') \times \nabla_{x'} G(x,x') dS'.
\end{equation}
We assume flat subdomains, which are excited by plane waves of the form
\begin{equation}
	\mathbf{H}(x) = \mathbf{H}e^{-ik \hat{p} \cdot x}.
\end{equation}
The induced currents are therefore also plane waves of the form
\begin{equation}
	\mathbf{J}(x') = \mathbf{J}e^{-ik \mathbf{v} \cdot x'},
	\quad \mathbf{J} = 2\hat{n}\times\mathbf{H},
	\quad \mathbf{v} = \hat{p} - \left( \hat{n}\cdot\hat{p} \right)\hat{n}.
\end{equation}
The integral over the subdomain of one of these currents is
\begin{equation}
	\int_{\Omega_n}
	\mathbf{J} \times \hat{R}
	\left( ik + \frac{1}{R} \right)
	\frac{e^{-ik\left(R + \mathbf{v} \cdot x'\right)}}{4\pi R} dS',
	\quad R = \norm{x-x'}
\end{equation}



At high frequencies the integral can be approximated by the stationary phase approximation,
which in two dimensions is given by
\begin{equation}
	\int g(x) e^{ikf(x)}
	\approx \frac{2\pi}{k} g(x^*)
	\abs{\det \nabla^2 f(x^*)}^{-1/2}
	e^{ikf(x^*) + \frac{i\pi}{4}\text{sign}\nabla^2 f(x^*)}
\end{equation}
for
\begin{equation}
	\nabla f(x^*) = 0.
\end{equation}
To apply this, we set
\begin{equation}
	g(x') = 
	\mathbf{J} \times \hat{R}
	\left( ik + \frac{1}{R} \right)
	\frac{1}{4\pi R}
\end{equation}
and
\begin{equation}
	f(x') = -\norm{x-x'} - \mathbf{v}\cdot x'.
\end{equation}

To find the stationary point of the phase function we first parameterize the phase function
in terms of variables on the surface of the subdomain. This parameterization is of the form
\begin{equation}
	x' = Jy' + x - h\hat{n},
\end{equation}
where $J$ is a unitary matrix of two vectors on the surface of the subdomain
and $h$ is the height of $x$ above the surface.
The parameterized phase function is
\begin{equation}
	f(y') = -\norm{-Jy' + h\hat{n}} - \mathbf{v}^T Jy' - \mathbf{v}^T x
	= - \sqrt{ y'^Ty' + h^2 } - \mathbf{v}^T Jy' - \mathbf{v}^T x.
\end{equation}
The gradient is
\begin{equation}
	\nabla_{y'} f(y') = 
	\begin{bmatrix}
		\frac{-y_1'}{\sqrt{y_1^{'2} + y_2^{'2} + h^2}} - (\mathbf{v}^TJ)_1 \\
		\frac{-y_2'}{\sqrt{y_1^{'2} + y_2^{'2} + h^2}} - (\mathbf{v}^TJ)_2
	\end{bmatrix}.
\end{equation}
The stationary point is therefore
\begin{equation}
	y^{'*} = \left( \frac{-h(\mathbf{v}^TJ)_1}{\sqrt{ 1 - \norm{\mathbf{v}}^2 }}, \frac{-h(\mathbf{v}^TJ)_2}{\sqrt{ 1 - \norm{\mathbf{v}}^2 }} \right)
	\quad\Rightarrow\quad x^{'*} = x - \frac{h}{\sqrt{ 1 - \norm{\mathbf{v}}^2 }}\mathbf{v} - h\hat{n},
\end{equation}
so
\begin{equation}
	R^* = \frac{h}{\sqrt{1-\norm{v}^2}},
	\quad \hat{R}^* - \left( \hat{n}\cdot\hat{R}^* \right)\hat{n} = \mathbf{v}
	\quad\Rightarrow\quad \hat{R}^* =  \hat{p}\text{, } \hat{p} - 2\left( \hat{n}\cdot\hat{p} \right)\hat{n},
\end{equation}
which is consistent with Snell's law.

\end{document}
