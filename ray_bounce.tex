\documentclass{article}

\usepackage[a4paper, total={6in, 8in}]{geometry}
\usepackage{amsmath}
\usepackage{amsfonts}
\usepackage{amsthm}
\usepackage{indentfirst}
\usepackage{hyperref}

\usepackage{algorithm}
\usepackage{algpseudocode}

\newcommand{\norm}[1]{\left\lVert #1 \right\rVert}
\newcommand{\abs}[1]{\left\lvert #1 \right\rvert}
\newcommand{\Div}[0]{\nabla\cdot}
\newcommand{\Curl}[0]{\nabla\times}

\newtheorem{theorem}{Theorem}[section]
\newtheorem{corollary}{Corollary}[theorem]
\newtheorem{lemma}[theorem]{Lemma}

\theoremstyle{plain}
\newtheorem*{remark}{Remark}

\title{Scattered field from plane wave on infinite flat plate}
\author{Ian Holloway}
\date{September 2024}

\begin{document}
\maketitle

\section{Introduction}\label{sec_intro}

The scattered magnetic field is
\begin{equation}
	\mathbf{H}(x) = \int \mathbf{J}(x') \times \nabla_{x'} G(x,x') dS'
\end{equation}
\begin{equation}
	\mathbf{H}(x)
	= \int \mathbf{J}(x') \times \nabla_{x'} \frac{e^{-ik\norm{x-x'}}}{4\pi\norm{x-x'}} dS'
\end{equation}
\begin{equation}
	\mathbf{H}(x)
	= -\int \mathbf{J}(x') \times \hat{R} \left( ik + \frac{1}{R} \right) \frac{e^{-ik\norm{x-x'}}}{4\pi\norm{x-x'}} dS'
\end{equation}
\begin{equation}
	\mathbf{H}(x)
	= -\int \mathbf{J}(x') \times \frac{x-x'}{\norm{x-x'}}
	\left( ik + \frac{1}{\norm{x-x'}} \right)
	\frac{e^{-ik\norm{x-x'}}}{4\pi\norm{x-x'}} dS'
\end{equation}
Assume that the surface is flat and is WLOG aligned with the $x_1x_2$ plane.
\begin{equation}
	\mathbf{H}(x)
	= -\int_{\mathbb{R}^2} \mathbf{J}(x') \times \frac{x-x'}{\norm{x-x'}}
	\left( ik + \frac{1}{\norm{x-x'}} \right)
	\frac{e^{-ik\norm{x-x'}}}{4\pi\norm{x-x'}} dx_1'dx_2'
\end{equation}
Let the current be a plane wave.
\begin{equation}
	\mathbf{J}(x') = \mathbf{J}e^{-ik(ax_1' + bx_2')}, \quad a^2 + b^2 \leq 1
\end{equation}
Now,
\begin{equation}
	\mathbf{H}(x)
	= -\mathbf{J} \times \int_{\mathbb{R}^2}
	\frac{x-x'}{\norm{x-x'}}
	\left( ik + \frac{1}{\norm{x-x'}} \right)
	\frac{1}{4\pi\norm{x-x'}}
	e^{-ik\left( \norm{x-x'} + ax_1' + bx_2' \right)} dx_1'dx_2'
\end{equation}


The stationary phase approximation in two dimensions is
\begin{equation}
	\int g(x) e^{ikf(x)}
	\approx \frac{2\pi}{k} g(x^*)
	\abs{\det \nabla^2 f(x^*)}^{-1/2}
	e^{ikf(x^*) + \frac{i\pi}{4}\text{sign}\nabla^2 f(x^*)}
\end{equation}
for
\begin{equation}
	\nabla f(x^*) = 0
\end{equation}


To apply this approximation to the radiation intergral,
we set
\begin{equation}
	g(x') = 
	\frac{x-x'}{\norm{x-x'}}
	\left( ik + \frac{1}{\norm{x-x'}} \right)
	\frac{1}{4\pi\norm{x-x'}}
\end{equation}
and
\begin{equation}
	f(x') = -\norm{x-x'} - ax_1' - bx_2'
\end{equation}
with
\begin{equation}
	\nabla_{x'} f(x') =
	\begin{bmatrix}
		\frac{x_1 - x_1'}{\norm{x-x'}} - a \\
		\frac{x_2 - x_2'}{\norm{x-x'}} - b
	\end{bmatrix}
\end{equation}
\begin{equation} % TODO
	\nabla^2_{x'} f(x') =
	\begin{bmatrix}
		\frac{x_1 - x_1'}{\norm{x-x'}} - a & ... \\
		\frac{x_2 - x_2'}{\norm{x-x'}} - b & ...
	\end{bmatrix}
\end{equation}

To find the stationary point of the radiation intergral, first WLOG let $x=(0,0,h)$,
so the gradient is
\begin{equation}
	\nabla_{x'} f(x') = 
	\begin{bmatrix}
		\frac{-x_1'}{\sqrt{x_1^{'2} + x_2^{'2} + h^2}} - a \\
		\frac{-x_2'}{\sqrt{x_1^{'2} + x_2^{'2} + h^2}} - b
	\end{bmatrix}
\end{equation} % note <x_1',x_2'> = c * <a,b>, then solve for c
We then get
\begin{equation}
	x^{'*} = \left( \frac{-ha}{\sqrt{ 1 - a^2 - b^2 }}, \frac{-hb}{\sqrt{ 1 - a^2 - b^2 }}, 0 \right)
\end{equation}


The approximated radiation integral is
\begin{equation} % TODO
	\mathbf{H}(x) \approx
	-\frac{2\pi}{k}
	\mathbf{J} \times \hat{R}^*
	\left( ik + \frac{1}{R^*} \right)
	\frac{1}{4\pi R^*}
	\left( \frac{1}{R^{*2}} \right)^{-1/2}
	e^{-ikR^*}
	e^{-ik(ax_1^{'*} + bx_2^{'*})}
	e^{\frac{i\pi}{4}\text{sign}\nabla^2 f(x^*)}
\end{equation}

\begin{equation} % TODO
	=
	-\frac{1}{2k}
	\mathbf{J}(x^{'*}) \times \hat{R}^*
	\left( ik + \frac{1}{R^*} \right)
	e^{-ikR^*}
	e^{\frac{i\pi}{4}\text{sign}\nabla^2 f(x^*)}
\end{equation}

\begin{equation} % TODO
	=
	-i\left( 1 + \frac{1}{ikR^*} \right)
	\left( \hat{n} \times \mathbf{H}_i(x^{'*}) \right) \times \hat{R}^*
	e^{-ik\hat{R}^*\cdot x}
	e^{\frac{i\pi}{4}\text{sign}\nabla^2 f(x^*)}
\end{equation}

\begin{equation} % TODO
	\approx
	-i
	\left( \hat{n} \times \mathbf{H}_i(x^{'*}) \right) \times \hat{R}^*
	e^{-ik\hat{R}^*\cdot x}
	e^{\frac{i\pi}{4}\text{sign}\nabla^2 f(x^*)}
\end{equation}





which is consistent with Snell's law
\begin{equation}
	\hat{R} - \left( \hat{n}\cdot\hat{R} \right)\hat{n} = 
	\begin{bmatrix}
		a \\
		b \\
		0
	\end{bmatrix}
\end{equation}



\end{document}
