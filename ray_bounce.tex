\documentclass{article}

\usepackage[a4paper, total={6in, 8in}]{geometry}
\usepackage{amsmath}
\usepackage{amsfonts}
\usepackage{amsthm}
\usepackage{indentfirst}
\usepackage{hyperref}

\usepackage{algorithm}
\usepackage{algpseudocode}

\newcommand{\norm}[1]{\left\lVert #1 \right\rVert}
\newcommand{\abs}[1]{\left\lvert #1 \right\rvert}
\newcommand{\Div}[0]{\nabla\cdot}
\newcommand{\Curl}[0]{\nabla\times}

\newtheorem{theorem}{Theorem}[section]
\newtheorem{corollary}{Corollary}[theorem]
\newtheorem{lemma}[theorem]{Lemma}

\theoremstyle{plain}
\newtheorem*{remark}{Remark}

\title{Scattered field from plane wave on infinite flat plate}
\author{Ian Holloway}
\date{September 2024}

\begin{document}
\maketitle

\section{Introduction}\label{sec_intro}

The scattered magnetic field is
\begin{equation}
	\mathbf{H}(x) = \int \mathbf{J}(x') \times \nabla_{x'} G(x,x') dS'
\end{equation}
\begin{equation}
	\mathbf{H}(x)
	= \int \mathbf{J}(x') \times \nabla_{x'} \frac{e^{-ik\norm{x-x'}}}{4\pi\norm{x-x'}} dS'
\end{equation}
\begin{equation}
	\mathbf{H}(x)
	= -\int \mathbf{J}(x') \times \hat{R} \left( ik + \frac{1}{R} \right) \frac{e^{-ik\norm{x-x'}}}{4\pi\norm{x-x'}} dS'
\end{equation}
\begin{equation}
	\mathbf{H}(x)
	= -\int \mathbf{J}(x') \times \frac{x-x'}{\norm{x-x'}}
	\left( ik + \frac{1}{\norm{x-x'}} \right)
	\frac{e^{-ik\norm{x-x'}}}{4\pi\norm{x-x'}} dS'
\end{equation}
Assume that the surface is flat and is WLOG aligned with the $x_1x_2$ plane.
\begin{equation}
	\mathbf{H}(x)
	= -\int_{\mathbb{R}^2} \mathbf{J}(x') \times \frac{x-x'}{\norm{x-x'}}
	\left( ik + \frac{1}{\norm{x-x'}} \right)
	\frac{e^{-ik\norm{x-x'}}}{4\pi\norm{x-x'}} dx_1'dx_2'
\end{equation}
Let the incident field be a plane wave
\begin{equation}
	\mathbf{H}_i = \mathbf{H}e^{-ik \hat{p} \cdot x},
\end{equation}
which induces a plane wave current
\begin{equation}
	\mathbf{J}(x') = \mathbf{J}e^{-ik(ax_1' + bx_2')}
\end{equation}
with
\begin{equation}
	\mathbf{J} = 2\hat{n}\times\mathbf{H},
	\quad \begin{bmatrix} a & b & 0 \end{bmatrix} = \hat{p} - \left( \hat{n}\cdot\hat{p} \right)\hat{n}.
\end{equation}
Now,
\begin{equation}
	\mathbf{H}(x)
	= -\mathbf{J} \times \int_{\mathbb{R}^2}
	\frac{x-x'}{\norm{x-x'}}
	\left( ik + \frac{1}{\norm{x-x'}} \right)
	\frac{1}{4\pi\norm{x-x'}}
	e^{-ik\left( \norm{x-x'} + ax_1' + bx_2' \right)} dx_1'dx_2'
\end{equation}


The stationary phase approximation in two dimensions is
\begin{equation}
	\int g(x) e^{ikf(x)}
	\approx \frac{2\pi}{k} g(x^*)
	\abs{\det \nabla^2 f(x^*)}^{-1/2}
	e^{ikf(x^*) + \frac{i\pi}{4}\text{sign}\nabla^2 f(x^*)}
\end{equation}
for
\begin{equation}
	\nabla f(x^*) = 0.
\end{equation}
To apply this approximation to the radiation intergral,
we set
\begin{equation}
	g(x') = 
	\frac{x-x'}{\norm{x-x'}}
	\left( ik + \frac{1}{\norm{x-x'}} \right)
	\frac{1}{4\pi\norm{x-x'}}
\end{equation}
and
\begin{equation}
	f(x') = -\norm{x-x'} - ax_1' - bx_2'.
\end{equation}

To find the stationary point of the radiation intergral, first WLOG let $x=(0,0,h)$,
so the gradient of the phase function is
\begin{equation}
	\nabla_{x'} f(x') = 
	\begin{bmatrix}
		\frac{x_1 - x_1'}{\norm{x-x'}} - a \\
		\frac{x_2 - x_2'}{\norm{x-x'}} - b
	\end{bmatrix}
	=
	\begin{bmatrix}
		\frac{-x_1'}{\sqrt{x_1^{'2} + x_2^{'2} + h^2}} - a \\
		\frac{-x_2'}{\sqrt{x_1^{'2} + x_2^{'2} + h^2}} - b
	\end{bmatrix}
\end{equation} % note <x_1',x_2'> = c * <a,b>, then solve for c
We compute
\begin{equation}
	x^{'*} = \left( \frac{-ha}{\sqrt{ 1 - a^2 - b^2 }}, \frac{-hb}{\sqrt{ 1 - a^2 - b^2 }}, 0 \right)
\end{equation}
so
\begin{equation}
	R^* = \frac{h}{\sqrt{1-a^2-b^2}},
	\quad \hat{R}^* - \left( \hat{n}\cdot\hat{R}^* \right)\hat{n} = \hat{p} - \left( \hat{n}\cdot\hat{p} \right)\hat{n}
	\quad\Rightarrow\quad \hat{R}^* = \hat{p} - 2\left( \hat{n}\cdot\hat{p} \right)\hat{n},
\end{equation}
which is consistent with Snell's law.


The Hessian of the phase function at the stationary point is
\begin{equation}
	\nabla^2_{x'} f(x^{'*}) =
	\frac{1}{R}\left( \hat{R}\hat{R}^T - I \right) =
	\frac{\sqrt{1-a^2-b^2}}{h}
	\begin{bmatrix}
		a^2-1 & ab \\
		ab & b^2-1
	\end{bmatrix}
\end{equation}
with determinant
\begin{equation}
	\det \nabla^2 f(x^{'*})
	= \frac{1-a^2-b^2}{R^{*2}}
	= \frac{\left(\hat{n}\cdot\hat{R}^*\right)^2}{R^{*2}}
\end{equation}
and eigenvalues
\begin{equation}
	\lambda\left( \nabla^2 f(x^{'*}) \right) = 
	-1, -\left( 1-a^2-b^2 \right)
	\quad\Rightarrow\quad \text{sign}\nabla^2 f(x^{'*}) = -2.
\end{equation}


The approximated radiation integral is
\begin{equation}
	\mathbf{H}(x) \approx
	-\frac{2\pi}{k}
	\mathbf{J} \times \hat{R}^*
	\left( ik + \frac{1}{R^*} \right)
	\frac{1}{4\pi R^*}
	\left( \frac{\left(\hat{n}\cdot\hat{R}^*\right)^2}{R^{*2}} \right)^{-1/2}
	e^{-ikR^*}
	e^{-ik(ax_1^{'*} + bx_2^{'*})}
	e^{\frac{-i\pi}{2}}
\end{equation}
\begin{equation}
	=
	-\frac{1}{2\left(\hat{n}\cdot\hat{R}^*\right)}
	\left( 1 + \frac{1}{ikR^*} \right)
	\mathbf{J}(x^{'*}) \times \hat{R}^*
	e^{-ik\hat{R}^*\cdot (R^*\hat{R}^*)}
\end{equation}
\begin{equation}
	\approx
	-\frac{1}{2\left(\hat{n}\cdot\hat{R}^*\right)}
	\mathbf{J}(x^{'*}) \times \hat{R}^*
	e^{-ik\hat{R}^*\cdot (R^*\hat{R}^*)}
\end{equation}
\begin{equation}
	=
	-\frac{1}{\left(\hat{n}\cdot\hat{R}^*\right)}
	\left( \hat{n} \times \mathbf{H}_i(x^{'*}) \right) \times \hat{R}^*
	e^{-ik\hat{R}^*\cdot (R^*\hat{R}^*)}
\end{equation}
Note that this is a plane wave.
It travels in a single direction, $\hat{R}$, and
there is no leading factor of $1/R$.




\end{document}
