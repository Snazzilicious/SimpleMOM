\documentclass{article}

\usepackage[a4paper, total={6in, 8in}]{geometry}
\usepackage{amsmath}
\usepackage{amsfonts}
\usepackage{amsthm}
\usepackage{indentfirst}
\usepackage{hyperref}

\usepackage{algorithm}
\usepackage{algpseudocode}

\newcommand{\norm}[1]{\left\lVert #1 \right\rVert}
\newcommand{\abs}[1]{\left\lvert #1 \right\rvert}
\newcommand{\Div}[0]{\nabla\cdot}
\newcommand{\Curl}[0]{\nabla\times}
\newcommand{\pDiv}[0]{\nabla'\cdot}
\newcommand{\pCurl}[0]{\nabla'\times}

\newtheorem{theorem}{Theorem}[section]
\newtheorem{corollary}{Corollary}[theorem]
\newtheorem{lemma}[theorem]{Lemma}

\theoremstyle{plain}
\newtheorem*{remark}{Remark}

\title{Non-LTI Electromagnetic Scattering}
\author{Ian Holloway}
\date{December 2024}

\begin{document}
\maketitle


\section{Introduction}

Linear methods such as MOM based on the EFIE or MFIE have proven powerful for the analysis of EM system.
Recently however interest has grown in non-linear circuitry which violates some of the key assumptions of the linear methods.
A more general method of simulation is thus necessary.


\section{Formulation}

The proposed method is based on the Jefimenko equations
which are generalizations of Coulomb's Law and the Biot-Savart law
for time-depedant charge and current distributions.
These are
\begin{equation}
	\mathbf{E}(x,t) = \frac{1}{4\pi\epsilon_0} \int
	\left( \frac{\rho(x^{'},t)}{R^2}
	+ \frac{1}{cR}\frac{\partial\rho}{\partial t}(x^{'},t_r)
	\right) \hat{R}
	- \frac{1}{c^2R}\frac{\partial\mathbf{J}}{\partial t}(x^{'},t_r)
	\,dx^{'}
\end{equation}
and
\begin{equation}
	\mathbf{B}(x,t) = -\frac{\mu_0}{4\pi} \int
	\hat{R} \times \left(
	\frac{1}{R^2}\mathbf{J}(x^{'},t_r)
	+ \frac{1}{cR}\frac{\partial\mathbf{J}}{\partial t}(x^{'},t_r)
	\right) \,dx^{'},
\end{equation}
where $R=\norm{x-x^{'}}$, $\hat{R} = (x-x^{'})/R$, and $t_r=t-R/c$ is the retarded time.
The domain of integration is the surface of the non-linear scatterer,
just as is the case in MOM-IE.

To complete the method, expressions giving the time derivatives of $\rho$ and $\mathbf{J}$ are necessary.
By definition,
\begin{equation}
	\frac{\partial\rho}{\partial t} = -\Div\mathbf{J}.
\end{equation}
I propose that we use Newton's second law to produce an equation for $\frac{\partial\mathbf{J}}{\partial t}$.
That is
\begin{equation*}
	\frac{\partial}{\partial t}(\rho_m\mathbf{v})
	= \mathbf{F}
	- \Div(\rho_m\mathbf{v}\otimes\mathbf{v})
\end{equation*}
\begin{equation*}
	\frac{\partial}{\partial t}\left( \frac{m_e}{e}\rho\mathbf{v} \right)
	= \rho\mathbf{E} + \mathbf{J}\times\mathbf{B}
	- \Div\left(\frac{m_e}{e}\rho\mathbf{v}\otimes\mathbf{v}\right)
\end{equation*}
\begin{equation}
	\frac{\partial\mathbf{J}}{\partial t}
	= \frac{e}{m_e}\left(
	\rho\mathbf{E} + \mathbf{J}\times\mathbf{B}
	\right)
	- \Div\left(\frac{1}{\rho}\mathbf{J}\otimes\mathbf{J}\right).
\end{equation}

We now have
\begin{equation}
	\frac{\partial\rho}{\partial t} = L_\rho( x,t, \rho, \mathbf{J} )
	\text{ and }
	\frac{\partial\mathbf{J}}{\partial t} = L_{\mathbf{J}}( x,t, \rho, \mathbf{J} )
\end{equation}
with
\begin{equation}
	L_\rho( \rho, \mathbf{J} ) = -\Div\mathbf{J}.
\end{equation}
and
\begin{multline}
	L_{\mathbf{J}}( \rho, \mathbf{J} ) = \\
	\frac{e}{m_e}\left[
	\frac{\rho}{4\pi\epsilon_0} \int
	\left( \frac{\rho(x^{'},t)}{R^2}
	+ \frac{1}{cR}\frac{\partial\rho}{\partial t}(x^{'},t_r)
	\right) \hat{R}
	- \frac{1}{c^2R}\frac{\partial\mathbf{J}}{\partial t}(x^{'},t_r)
	\,dx^{'}\right. \\
	\left.
	-\mathbf{J}\times
	\frac{\mu_0}{4\pi} \int
	\hat{R} \times \left(
	\frac{1}{R^2}\mathbf{J}(x^{'},t_r)
	+ \frac{1}{cR}\frac{\partial\mathbf{J}}{\partial t}(x^{'},t_r)
	\right) \,dx^{'}
	\right]\\
	+ \frac{e}{m_e}\left(
	\rho\mathbf{E}_i(x,t) + \mathbf{J}\times\mathbf{B}_i(x,t)
	\right) \\
	- \Div\left(\frac{1}{\rho}\mathbf{J}\otimes\mathbf{J}\right).
\end{multline}
Here $\mathbf{E}_i(x,t)$ and $\mathbf{B}_i(x,t)$ are the imposed electric and magnetic fields.


\subsection{Far Field}

Once we know the charge and current distributions at relevant times,
we can compute far field.
We define the quantity of interest as
\begin{equation}
	\mathbf{E}_\infty(\hat{R},t) = \lim_{r\rightarrow\infty} r\abs{\mathbf{E}(r\hat{R},t)}.
\end{equation}

Inserting the integral into this we get
\begin{equation}
	\mathbf{E}_\infty(\hat{R},t) = \frac{1}{4\pi\epsilon_0} \int
	\frac{1}{c}\frac{\partial\rho}{\partial t}(x^{'},t_r) \hat{R}
	- \frac{1}{c^2}\frac{\partial\mathbf{J}}{\partial t}(x^{'},t_r)
	\,dx^{'},
\end{equation}
where the retarded time is now given by $t_r = t - \hat{R}\cdot x^{'}$.


\section{Algorithm}

% TODO save history



\end{document}
