\documentclass{article}

\usepackage[a4paper, total={6in, 8in}]{geometry}
\usepackage{amsmath}
\usepackage{amsfonts}
\usepackage{amsthm}
\usepackage{indentfirst}
\usepackage{hyperref}

\newcommand{\norm}[1]{||#1||}
\newcommand{\Div}[0]{\nabla\cdot}
\newcommand{\Curl}[0]{\nabla\times}

\theoremstyle{plain}
\newtheorem*{remark}{Remark}

\begin{document}




% want to get error bound for a given frequency - hopefully can integrate e^ikS'(x-x0) analytically |I| <= 2/k, also f'(x-x0)e^ikS'(x-x0)
% state conditions for integral to be approximately zero \phi >> 1/k >> grad f \cdot grad \phi
% TODO
% Set up rectangular patch - for fourier series in particular
% include expression for stationary point


\section{Intro}\label{sec_intro}



\section{Taylor Series}


% pick domain - start on the manifold
% TODO describe the problem set up better
Consider the integral
\begin{equation}
	I = \int\int f(x) e^{ikS(x)} \sqrt{g} dx_1dx_2.
\end{equation}
For convenience, we absorb the metric into $f$.
\begin{equation}
	I = \int\int F(x) e^{ikS(x)} dx_1dx_2
\end{equation}



% pick x0 (to evaluate grad S) - also an x0 to evaluate grad F (doesn't need to be the same x0)
% taylor expand
We suppose that both $F$ and $S$ can be approximated well by a Taylor series expansion about $x^*$.
Thus
\begin{equation}
	I \approx \int\int \left[ F(x^*) + \nabla F(x^*) \cdot (x-x^*) \right] e^{ik\left[S(x^*) + \nabla S(x^*)\cdot (x-x^*) \right]} dx_1dx_2.
\end{equation}
We assume that $\frac{dS}{dx_2}(x^*) = 0$ and consequently $\frac{dS}{dx_1}(x^*) = \norm{\nabla S(x^*)}$.
If this is not already the case, a new coordinate system can be introduced for which it is. See \ref{sec_rotation}.


% sort terms and integrate
We now have
\begin{equation}
	I \approx \int\int \left[ F(x^*) + \nabla F(x^*) \cdot (x-x^*) \right] e^{ik\left[S(x^*) + \norm{\nabla S(x^*)}(x_1-x_1*) \right]} dx_1dx_2.
\end{equation}
%\begin{multline}
%	I \approx F(x^*) e^{ikS(x^*)} \int\int e^{ik\norm{\nabla S(x^*)}(x_1-x_1^*)} dx_1dx_2 \\
%	+ e^{ikS(x^*)}\int\int \nabla F(x^*) \cdot (x-x^*) e^{ik\norm{\nabla S(x^*)}(x_1-x_1^*)} dx_1dx_2.
%\end{multline}
\begin{multline}
	= F(x^*) e^{ikS(x^*)} \int\int e^{ik\norm{\nabla S(x^*)}(x_1-x_1*)} dx_1dx_2 \\
	+ \frac{dF}{dx_1}(x^*) e^{ikS(x^*)} \int\int (x_1-x_1^*) e^{ik\norm{\nabla S(x^*)}(x_1-x_1^*)} dx_1dx_2 \\
	+ \frac{dF}{dx_2}(x^*) e^{ikS(x^*)} \int (x_2-x_2^*) \int e^{ik\norm{\nabla S(x^*)}(x_1-x_1^*)} dx_1dx_2.
\end{multline}


% work out integral(s)
% e^ikSx - note that (e^i\pi)^2n - 1 = 0
% x eikSx

Consider first the integral
\begin{equation}
	I_1 = \int_A^B e^{ik\norm{\nabla S(x^*)}(x_1-x_1*)} dx_1 = \frac{e^{ik\norm{\nabla S(x^*)}A}}{ik\norm{\nabla S(x^*)}} \left( e^{ik\norm{\nabla S(x^*)}(B-A)}-1 \right).
\end{equation}
This shows that
\begin{equation}
	|I_1| \leq \frac{2}{k\norm{\nabla S(x^*)}},
\end{equation}
independent of the domain of integration.
In particular, $|I_1| = 0$ for $B-A = 2n\pi / k\norm{\nabla S(x^*)}$, $n\in\mathbb{N}$.




\section{Fourier Transform}

% pick domain - start on the manifold
% TODO describe the problem set up better
Consider the integral on a manifold
\begin{equation}
	I = \int\int f(x) e^{ikS(x)} \sqrt{g} dx_1dx_2.
\end{equation}
For convenience, we absorb the metric into the function as  $F = f\sqrt{g}$.
\begin{equation}
	I = \int\int F(x) e^{ikS(x)} dx_1dx_2
\end{equation}



% pick x0 (to evaluate grad S) - also an x0 to evaluate grad F (doesn't need to be the same x0)
% taylor expand
We suppose that $S$ can be approximated well by a Taylor series expansion about $x^*$.
Thus
\begin{equation}
	I \approx \int\int F(x) e^{ik\left(S(x^*) + \nabla S(x^*)\cdot (x-x^*) \right)} dx_1dx_2.
\end{equation}
We assume that $\frac{dS}{dx_2}(x^*) = 0$ and consequently $\frac{dS}{dx_1}(x^*) = \norm{\nabla S(x^*)}$.
If this is not already the case, a new coordinate system can be introduced for which it is. See \ref{sec_rotation} for details.


% Fourier transform
We now replace $F$ with a Fourier series
\begin{equation}
	F(x) = \sum_n \alpha_n(x_2) e^{ \frac{i2\pi n}{ \Delta x_1}x_1 }
\end{equation}
with coefficients
\begin{equation}
	\alpha_n(x_2) = \int F(x) e^{ \frac{-i2\pi n}{ \Delta x_1}x_1 } dx_1
\end{equation}


% sort terms and integrate
We now have
\begin{equation}
	I \approx \sum_n \int\int \alpha_n(x_2) e^{ \frac{i2\pi n}{ \Delta x_1 }x_1 }
	e^{ik\left[S(x^*) + \norm{\nabla S(x^*)}(x_1-x_1^*) \right]} dx_1dx_2.
\end{equation}
\begin{equation}
	= \sum_n e^{ik \left( S(x^*) - \norm{\nabla S(x^*)}x_1^* \right)} 
	\int \alpha_n(x_2) dx_2 
	\int e^{ i\left( k\norm{\nabla S(x^*)} + \frac{2\pi n}{ \Delta x_1 }\right) x_1 } dx_1
\end{equation}


% work out integral(s)
% e^ikSx - note that (e^i\pi)^2n - 1 = 0
% x eikSx

The integral over $x_1$ is
\begin{equation}
	\int_a^b e^{ i\left( k\norm{\nabla S(x^*)} + \frac{2\pi n}{ \Delta x_1 }\right) x_1 } dx_1
	= \frac{e^{ i\left( k\norm{\nabla S(x^*)} + \frac{2\pi n}{ \Delta x_1 }\right) a } }{ ik\norm{\nabla S(x^*)} + \frac{i2\pi n}{ \Delta x_1 } }
	\left( e^{ i\left( k\norm{\nabla S(x^*)}\Delta x_1 + 2\pi n \right)  } - 1 \right).
\end{equation}
This shows that
\begin{equation}
	|I_1| \leq \frac{2}{k\norm{\nabla S(x^*)}},
\end{equation}
independent of the domain of integration.
In particular, $|I_1| = 0$ for $B-A = 2n\pi / k\norm{\nabla S(x^*)}$, $n\in\mathbb{N}$.






\section{Coordinate rotation}\label{sec_rotation}
% rotate coordinates to align with grad S at expansion point
%	include grad S in rotation matrix

























\end{document}



