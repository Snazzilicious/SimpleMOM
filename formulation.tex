\documentclass{article}

\usepackage[a4paper, total={6in, 8in}]{geometry}
\usepackage{amsmath}
\usepackage{amsfonts}
\usepackage{amsthm}
\usepackage{indentfirst}
\usepackage{hyperref}

\newcommand{\abs}[1]{\left\lvert#1\right\rvert}
\newcommand{\norm}[1]{\left\lVert#1\right\rVert}
\newcommand{\Div}[0]{\nabla\cdot}
\newcommand{\Curl}[0]{\nabla\times}

\theoremstyle{plain}
\newtheorem*{remark}{Remark}

\begin{document}




% want to get error bound for a given frequency - hopefully can integrate e^ikS'(x-x0) analytically |I| <= 2/k, also f'(x-x0)e^ikS'(x-x0)
% state conditions for integral to be approximately zero \phi >> 1/k >> grad f \cdot grad \phi
% TODO
% Set up rectangular patch - for fourier series in particular
% include expression for stationary point


\section{Intro}\label{sec_intro}



\section{Stationary Phase}\label{sec_stationaryPhase}

% 1D example
% include expression for stationary point from Pathak or wikipedia

The goal of this section is to show under what conditions an integral involving a
highly oscillatory term produces an almost zero result.
% TODO include those conditions here

Consider the integral on a manifold
\begin{equation}
    I = \int_\Omega f e^{ikS} \sqrt{g} \,dA,
\end{equation}
in which $f$, $S$, and $g$ are functions of $x$, with dependence suppressed for simplicity.
The domain $\Omega$ is assumed simple, meaning that any line through it intersects the boundary in exactly 2 places.
For convenience, we absorb the square root of the metric determinant into $f$.
\begin{equation}
    I = \int_\Omega F e^{ikS} \,dA
\end{equation}
rewrite
\begin{equation}
    = \int_\Omega \frac{F}{ik\mathbf{v}\cdot\nabla S} \nabla\cdot\left( \mathbf{v}e^{ikS} \right) \,dA
\end{equation}
Green's Identity
\begin{equation}
    = \int_{d\Omega} \frac{Fe^{ikS}}{ik\mathbf{v}\cdot\nabla S} \mathbf{v}\cdot\hat{n} \,dL
    - \int_\Omega e^{ikS} \mathbf{v}\cdot\nabla\left( \frac{F}{ik\mathbf{v}\cdot\nabla S} \right) \,dA
\end{equation}
Product rule and note that $\mathbf{v}\cdot\nabla\left(\mathbf{v}\cdot\nabla S\right) = \mathbf{v}^TH_S\mathbf{v}$
\begin{equation}
    = \int_{d\Omega} \frac{Fe^{ikS}}{ik\mathbf{v}\cdot\nabla S} \mathbf{v}\cdot\hat{n} \,dL
    - \int_\Omega \frac{ e^{ikS} }{ {ik\mathbf{v}\cdot\nabla S} } \mathbf{v}\cdot\nabla F
    - Fe^{ikS} \frac{\mathbf{v}^TH_S\mathbf{v}}{ ik\left(\mathbf{v}\cdot\nabla S\right)^2} \,dA
\end{equation}
% TODO move last term over to original integral and argue that the factor is small
% TODO use Taylor series to replace integrand on d\Omega with a value where the limits of integration are a multiple of 2\pi
\begin{equation}
    \abs{ \int_\Omega \frac{ e^{ikS} }{ {ik\mathbf{v}\cdot\nabla S} } \mathbf{v}\cdot\nabla F \,dA }
    \leq \frac{\abs{\Omega}}{k} \frac{\max\abs{\mathbf{v}\cdot\nabla F}}{\min\abs{\mathbf{v}\cdot\nabla S}}
\end{equation}


% TODO
% set up relationship among dL, \mathbf{t}, and \hat{n}

We separate the boundary into two arcs, $\Gamma^+$ and $\Gamma^-$,
which are respectively the portions of the domain where $\mathbf\cdot\hat{n} > 0$ and $\mathbf\cdot\hat{n} < 0$.
The arcs are parameterized by the variable $\xi$ in such a way that
\begin{equation}
	\Gamma^+(\xi) - \Gamma^-(\xi) = \alpha(\xi)\mathbf{v},
\end{equation}
where $\alpha(\xi)$ is a scalar function.
The tangent vectors for the arcs are given by
\begin{equation}
	\mathbf{t}^\pm = \frac{d\Gamma}{d\xi}^\pm.
\end{equation}
We thus get
\begin{equation}
	\mathbf{t}^+(\xi) - \mathbf{t}^-(\xi) = \frac{d\alpha}{d\xi}(\xi)\mathbf{v}.
\end{equation}
Multiplying by the rotation matrix we get
\begin{equation}
	\mathbf{n}^+(\xi) + \mathbf{n}^-(\xi) = \frac{d\alpha}{d\xi}(\xi)R\mathbf{v}.
\end{equation}
We now multiply by $\mathbf{v}$ to get
\begin{equation}
	\mathbf{v}\cdot\mathbf{n}^+(\xi) + \mathbf{v}\cdot\mathbf{n}^-(\xi) = 0,
\end{equation}
and so finally
\begin{equation}
	\mathbf{v}\cdot\mathbf{n}^+(\xi) = -\mathbf{v}\cdot\mathbf{n}^-(\xi).
\end{equation}
The path integral is now given by
\begin{equation}
	 \int \left( \frac{Fe^{ikS}}{ik\mathbf{v}\cdot\nabla S} \mathbf{v}\cdot\hat{n}
	 - \frac{Fe^{ikS}}{ik\mathbf{v}\cdot\nabla S} \mathbf{v}\cdot\hat{n} \right)\mathbf{v}\cdot\mathbf{n} \,d\xi
\end{equation}



\section{Taylor Series}


% pick domain - start on the manifold
% TODO describe the problem set up better
Consider the integral
\begin{equation}
	I = \int\int f(x) e^{ikS(x)} \sqrt{g} dx_1dx_2.
\end{equation}
For convenience, we absorb the metric into $f$.
\begin{equation}
	I = \int\int F(x) e^{ikS(x)} dx_1dx_2
\end{equation}



% pick x0 (to evaluate grad S) - also an x0 to evaluate grad F (doesn't need to be the same x0)
% taylor expand
We suppose that both $F$ and $S$ can be approximated well by a Taylor series expansion about $x^*$.
Thus
\begin{equation}
	I \approx \int\int \left[ F(x^*) + \nabla F(x^*) \cdot (x-x^*) \right] e^{ik\left[S(x^*) + \nabla S(x^*)\cdot (x-x^*) \right]} dx_1dx_2.
\end{equation}
We assume that $\frac{dS}{dx_2}(x^*) = 0$ and consequently $\frac{dS}{dx_1}(x^*) = \norm{\nabla S(x^*)}$.
If this is not already the case, a new coordinate system can be introduced for which it is. See \ref{sec_rotation}.


% sort terms and integrate
We now have
\begin{equation}
	I \approx \int\int \left[ F(x^*) + \nabla F(x^*) \cdot (x-x^*) \right] e^{ik\left[S(x^*) + \norm{\nabla S(x^*)}(x_1-x_1*) \right]} dx_1dx_2.
\end{equation}
%\begin{multline}
%	I \approx F(x^*) e^{ikS(x^*)} \int\int e^{ik\norm{\nabla S(x^*)}(x_1-x_1^*)} dx_1dx_2 \\
%	+ e^{ikS(x^*)}\int\int \nabla F(x^*) \cdot (x-x^*) e^{ik\norm{\nabla S(x^*)}(x_1-x_1^*)} dx_1dx_2.
%\end{multline}
\begin{multline}
	= F(x^*) e^{ikS(x^*)} \int\int e^{ik\norm{\nabla S(x^*)}(x_1-x_1*)} dx_1dx_2 \\
	+ \frac{dF}{dx_1}(x^*) e^{ikS(x^*)} \int\int (x_1-x_1^*) e^{ik\norm{\nabla S(x^*)}(x_1-x_1^*)} dx_1dx_2 \\
	+ \frac{dF}{dx_2}(x^*) e^{ikS(x^*)} \int (x_2-x_2^*) \int e^{ik\norm{\nabla S(x^*)}(x_1-x_1^*)} dx_1dx_2.
\end{multline}


% work out integral(s)
% e^ikSx - note that (e^i\pi)^2n - 1 = 0
% x eikSx

Consider the integral
\begin{equation}
	I_1 = \int_A^B e^{ik\norm{\nabla S(x^*)}(x_1-x_1*)} dx_1 
	= \frac{e^{ik\norm{\nabla S(x^*)}(A-x^*)}}{ik\norm{\nabla S(x^*)}}
	\left( e^{ik\norm{\nabla S(x^*)}(B-A)}-1 \right).
\end{equation}
This shows that
\begin{equation}
	|I_1| \leq \frac{2}{k\norm{\nabla S(x^*)}},
\end{equation}
independent of the domain of integration.
In particular, $|I_1| = 0$ for $B-A = 2n\pi / k\norm{\nabla S(x^*)}$, $n\in\mathbb{N}$.

Now consider the integral 
\begin{equation*}
	I_2 = \int_A^B (x_1-x_1^*) e^{ik\norm{\nabla S(x^*)}(x_1-x_1^*)} dx_1
\end{equation*}
\begin{equation}
	= \left[ \frac{(x_1-x_1^*)}{ik\norm{\nabla S(x^*)}} e^{ik\norm{\nabla S(x^*)}(x_1-x_1^*)} \right]_A^B
	- \frac{1}{ik\norm{\nabla S(x^*)}}\int_A^B e^{ik\norm{\nabla S(x^*)}(x_1-x_1^*)} dx_1
\end{equation}



\section{Fourier Transform}

% pick domain - start on the manifold
% TODO describe the problem set up better
Consider the integral on a manifold
\begin{equation}
	I = \int\int f(x) e^{ikS(x)} \sqrt{g} dx_1dx_2.
\end{equation}
For convenience, we absorb the metric into the function as  $F = f\sqrt{g}$.
\begin{equation}
	I = \int\int F(x) e^{ikS(x)} dx_1dx_2
\end{equation}



% pick x0 (to evaluate grad S) - also an x0 to evaluate grad F (doesn't need to be the same x0)
% taylor expand
We suppose that $S$ can be approximated well by a Taylor series expansion about $x^*$.
Thus
\begin{equation}
	I \approx \int\int F(x) e^{ik\left(S(x^*) + \nabla S(x^*)\cdot (x-x^*) \right)} dx_1dx_2.
\end{equation}
We assume that $\frac{dS}{dx_2}(x^*) = 0$ and consequently $\frac{dS}{dx_1}(x^*) = \norm{\nabla S(x^*)}$.
If this is not already the case, a new coordinate system can be introduced for which it is. See \ref{sec_rotation} for details.


% Fourier transform
We now replace $F$ with a Fourier series
\begin{equation}
	F(x) = \sum_n \alpha_n(x_2) e^{ \frac{i2\pi n}{ \Delta x_1}x_1 }
\end{equation}
with coefficients
\begin{equation}
	\alpha_n(x_2) = \int F(x) e^{ \frac{-i2\pi n}{ \Delta x_1}x_1 } dx_1
\end{equation}


% sort terms and integrate
We now have
\begin{equation}
	I \approx \sum_n \int\int \alpha_n(x_2) e^{ \frac{i2\pi n}{ \Delta x_1 }x_1 }
	e^{ik\left[S(x^*) + \norm{\nabla S(x^*)}(x_1-x_1^*) \right]} dx_1dx_2.
\end{equation}
\begin{equation}
	= \sum_n e^{ik \left( S(x^*) - \norm{\nabla S(x^*)}x_1^* \right)} 
	\int \alpha_n(x_2) dx_2 
	\int e^{ i\left( k\norm{\nabla S(x^*)} + \frac{2\pi n}{ \Delta x_1 }\right) x_1 } dx_1
\end{equation}


% work out integral(s)
% e^ikSx - note that (e^i\pi)^2n - 1 = 0
% x eikSx

The integral over $x_1$ is
\begin{equation}
	\int_a^b e^{ i\left( k\norm{\nabla S(x^*)} + \frac{2\pi n}{ \Delta x_1 }\right) x_1 } dx_1
	= \frac{e^{ i\left( k\norm{\nabla S(x^*)} + \frac{2\pi n}{ \Delta x_1 }\right) a } }{ ik\norm{\nabla S(x^*)} + \frac{i2\pi n}{ \Delta x_1 } }
	\left( e^{ i\left( k\norm{\nabla S(x^*)}\Delta x_1 + 2\pi n \right)  } - 1 \right).
\end{equation}
This shows that
\begin{equation}
	|I_1| \leq \frac{2}{k\norm{\nabla S(x^*)} + \frac{2\pi n}{ \Delta x_1 }},
\end{equation}
independent of the domain of integration.
In particular, $|I_1| = 0$ for $B-A = 2n\pi / k\norm{\nabla S(x^*)}$, $n\in\mathbb{N}$.






\section{Coordinate rotation}\label{sec_rotation}
% rotate coordinates to align with grad S at expansion point
%	include grad S in rotation matrix

























\end{document}



