\documentclass{article}

\usepackage[a4paper, total={6in, 8in}]{geometry}
\usepackage{amsmath}
\usepackage{amsfonts}
\usepackage{amsthm}
\usepackage{indentfirst}
\usepackage{hyperref}

\usepackage{algorithm}
\usepackage{algpseudocode}

\newcommand{\norm}[1]{\left\lVert #1 \right\rVert}
\newcommand{\abs}[1]{\left\lvert #1 \right\rvert}
\newcommand{\Div}[0]{\nabla\cdot}
\newcommand{\Curl}[0]{\nabla\times}

\newtheorem{theorem}{Theorem}[section]
\newtheorem{corollary}{Corollary}[theorem]
\newtheorem{lemma}[theorem]{Lemma}

\theoremstyle{plain}
\newtheorem*{remark}{Remark}

\title{Electromagnetic Scattering}
\author{Ian Holloway}
\date{September 2024}

\begin{document}
\maketitle


\section{Derivation of PO-SBR from MFIE}

We start with the MFIE problem defined by
\begin{equation}
	2\hat{n}\times\mathbf{H}_i(x)
	= \mathbf{J}(x)
	- 2\hat{n}\times\int \nabla_{x} G(x,x') \times \mathbf{J}(x')  dS'.
\end{equation}
We then break the domain of integration into pieces
\begin{equation}
	2\hat{n}\times\mathbf{H}_i(x)
	= \mathbf{J}(x)
	- \sum_n 2\hat{n}\times\int_{\Omega_n} \nabla_{x} G(x,x') \times \mathbf{J}(x') dS'.
\end{equation}
We assume flat subdomains, which are excited by plane waves of the form
\begin{equation}
	\mathbf{H}(x) = \mathbf{H}e^{-ik \hat{p} \cdot x}.
\end{equation}
The induced currents are therefore also plane waves of the form
\begin{equation}
	\mathbf{J}(x') = \mathbf{J}e^{-ik \mathbf{v} \cdot x'},
	\quad \mathbf{J} = 2\hat{n}\times\mathbf{H},
	\quad \mathbf{v} = \hat{p} - \left( \hat{n}\cdot\hat{p} \right)\hat{n}.
\end{equation}
The integral over the subdomain of one of these currents is
\begin{equation}
	\int_{\Omega_n}
	\mathbf{J} \times \hat{R}
	\left( ik + \frac{1}{R} \right)
	\frac{e^{-ik\left(R + \mathbf{v} \cdot x'\right)}}{4\pi R} dS',
	\quad R = \norm{x-x'}
\end{equation}



At high frequencies the integral can be approximated by the stationary phase approximation,
which in two dimensions is given by
\begin{equation}
	\int g(x) e^{ikf(x)}
	\approx \frac{2\pi}{k} g(x^*)
	\abs{\det \nabla^2 f(x^*)}^{-1/2}
	e^{ikf(x^*) + \frac{i\pi}{4}\text{sign}\nabla^2 f(x^*)}
\end{equation}
for
\begin{equation}
	\nabla f(x^*) = 0.
\end{equation}
To apply this, we set
\begin{equation}
	g(x') = 
	\mathbf{J} \times \hat{R}
	\left( ik + \frac{1}{R} \right)
	\frac{1}{4\pi R}
\end{equation}
and
\begin{equation}
	f(x') = -\norm{x-x'} - \mathbf{v}\cdot x'.
\end{equation}

To find the stationary point of the phase function we first parameterize the phase function
in terms of variables on the surface of the subdomain. This parameterization is of the form
\begin{equation}
	x' = Jy' + x - h\hat{n},
\end{equation}
where $J$ is a unitary matrix of two vectors on the surface of the subdomain
and $h$ is the height of $x$ above the surface.
The parameterized phase function is
\begin{equation}
	f(y') = -\norm{-Jy' + h\hat{n}} - \mathbf{v}^T Jy' - \mathbf{v}^T x
	= - \sqrt{ y'^Ty' + h^2 } - \mathbf{v}^T Jy' - \mathbf{v}^T x.
\end{equation}
The gradient is
\begin{equation}
	\nabla_{y'} f(y') = 
	\begin{bmatrix}
		\frac{-y_1'}{\sqrt{y_1'^{2} + y_2'^{2} + h^2}} - (\mathbf{v}^TJ)_1 \\
		\frac{-y_2'}{\sqrt{y_1'^{2} + y_2'^{2} + h^2}} - (\mathbf{v}^TJ)_2
	\end{bmatrix}.
\end{equation} % note <y_1',y_2'> = c * v^TJ, then solve for c
The stationary point is therefore
\begin{equation}
	y'^{*} = \left( \frac{-h(\mathbf{v}^TJ)_1}{\sqrt{ 1 - \norm{\mathbf{v}}^2 }}, \frac{-h(\mathbf{v}^TJ)_2}{\sqrt{ 1 - \norm{\mathbf{v}}^2 }} \right)
	\quad\Rightarrow\quad x'^{*} = x - \frac{h}{\sqrt{ 1 - \norm{\mathbf{v}}^2 }}\mathbf{v} - h\hat{n},
\end{equation}
so
\begin{equation}
	R^* = \frac{h}{\sqrt{1-\norm{v}^2}},
	\quad \hat{R}^* - \left( \hat{n}\cdot\hat{R}^* \right)\hat{n} = \mathbf{v}
	\quad\Rightarrow\quad \hat{R}^* =  \hat{p}\text{, } \hat{p} - 2\left( \hat{n}\cdot\hat{p} \right)\hat{n},
\end{equation}
which is consistent with Snell's law.

The Hessian of the phase function is
\begin{equation}
	\nabla^2_{y'} f(y') =
	\frac{1}{ \left( y_1'^{2} + y_2'^{2} + h^2 \right)^{3/2}}
	\begin{bmatrix}
		-y_2'^{2} - h^2 & y_1'y_2' \\
		y_1'y_2' & -y_1'^{2} - h^2 
	\end{bmatrix}
\end{equation}
and at the stationary point
\begin{equation}
	\nabla^2_{y'} f(y'^{*}) =
	\frac{\sqrt{1-\norm{v}^2}}{h}
	\begin{bmatrix}
		(\mathbf{v}^TJ)_1^2 - 1 & (\mathbf{v}^TJ)_1(\mathbf{v}^TJ)_2 \\
		(\mathbf{v}^TJ)_1(\mathbf{v}^TJ)_2 & (\mathbf{v}^TJ)_2^2 - 1
	\end{bmatrix}
\end{equation}
with determinant
\begin{equation}
	\det \nabla^2 f(y'^{*})
	= \frac{1-\norm{\mathbf{v}}^2}{R^{*2}}
	= \frac{\left(\hat{n}\cdot\hat{R}^*\right)^2}{R^{*2}}
\end{equation}
and eigenvalues
\begin{equation}
	\lambda\left( \nabla^2 f(y'^{*}) \right) = 
	-1, -\left( 1-\norm{\mathbf{v}}^2 \right)
	\quad\Rightarrow\quad \text{sign}\nabla^2 f(y'^{*}) = -2.
\end{equation}



The approximated radiation integral is now 
\begin{equation}
	\mathbf{H}(x) \approx
	\frac{2\pi}{k}
	\mathbf{J} \times \hat{R}^*
	\left( ik + \frac{1}{R^*} \right)
	\frac{1}{4\pi R^*}
	\left( \frac{\left(\hat{n}\cdot\hat{R}^*\right)^2}{R^{*2}} \right)^{-1/2}
	e^{-ikR^*}
	e^{-ik\mathbf{v}\cdot x'^{*}}
	e^{\frac{-i\pi}{2}}
\end{equation}
\begin{equation}
	=
	\frac{1}{2\abs{\hat{n}\cdot\hat{R}^*}}
	\left( 1 + \frac{1}{ikR^*} \right)
	\mathbf{J}(x'^{*}) \times \hat{R}^*
	e^{-ik\hat{R}^*\cdot (R^*\hat{R}^*)}
\end{equation}
\begin{equation}
	\approx
	\frac{1}{2\abs{\hat{n}\cdot\hat{R}^*}}
	\mathbf{J}(x'^{*}) \times \hat{R}^*
	e^{-ik\hat{R}^*\cdot (R^*\hat{R}^*)}
\end{equation}
\begin{equation}
	=
	\frac{1}{\abs{\hat{n}\cdot\hat{R}^*}}
	\left( \hat{n} \times \mathbf{H}_i(x'^{*}) \right) \times \hat{R}^*
	e^{-ik\hat{R}^*\cdot (R^*\hat{R}^*)}
\end{equation}
Note that there is no leading factor of $1/R$,
making this behave more like a plane wave than a spherical wave.
Furthermore, when we plug in the incident wave direction for the scattered direction, 
$\hat{R}=\hat{p}$, 
and keep in mind that the amplitude of the incident wave is perpendicular to its propagation direction,
we get the scattered amplitude
\begin{equation}
	\frac{1}{\abs{\hat{n}\cdot\hat{R}^*}}
	\left( \hat{n} \times \mathbf{H}_i(x'^{*}) \right) \times \hat{R}^*
	=
	\frac{1}{\abs{\hat{n}\cdot\hat{p}}}
	\left( \left(\hat{n}\cdot\hat{p}\right)\mathbf{H}_i(x'^{*}) - \left( \hat{p}\cdot\mathbf{H}_i(x'^{*}) \right)\hat{n} \right)
	= -\mathbf{H}_i(x'^{*})
\end{equation}
which perfectly cancels the incident wave.
The surface therefore can be treated as impenetrable.


With this result we can simplify the integral equations.
First by inserting the approximated form of the integral over each subdomain,
then by restricting the summation to only be over terms for which there is an
unobstructed path $\hat{R}$ from the subdomain to the test location.
\begin{equation}
	2\hat{n}\times\mathbf{H}_i(x)
	= \mathbf{J}(x)
	- \sum_{n\in V_x} 2\hat{n}\times
	\frac{1}{2\left(\hat{n}\cdot\hat{R}^*_n\right)}
	\mathbf{J}(x'^{*}_n) \times \hat{R}^*_n
	e^{-ik\hat{R}^*_n\cdot (R^*_n\hat{R}^*_n)}
\end{equation}

An iterative algorithm naturally follows to compute the total current $\mathbf{J}(x)$ that solves this system of equations.
It begins by decomposing the incident excitation into rays.
These rays are traced to see if and where they strike the surface of the scatterer.
Whenever rays strike the surface, currents are computed via $\mathbf{J}=2\hat{n}\times \mathbf{H}_i$
and stored for radiating later.
Then the reflected direction is computed via $\hat{R} = \hat{p} - 2(\hat{n}\cdot\hat{p})\hat{n}$
and a reflected amplitude is computed via $\mathbf{H} = \frac{1}{2\left(\hat{n}\cdot\hat{R}^*\right)}\mathbf{J} \times \hat{R}$.
A reflected ray is computed with these values and is saved in a queue to be the input of this procedure again.
The procedure is performed repeatedly until no new rays are created or the currents are sufficiently converged.
The currents can then be radiated to the farfield. This algorithm is pseudocoded in Algorithm \ref{alg:posbr}.
\begin{algorithm}
\caption{PO-SBR}\label{alg:posbr}
\begin{algorithmic}[1]
\State $currents \gets \emptyset$
\State $rayBuffer \gets \emptyset$
\State $rayBuffer.push( excitation.rays )$
\While{$rayBuffer.size > 0$}
\State $ray \gets rayBuffer.pop()$
\State $facet \gets trace(ray)$
\If{$facet$ is not None}
    \State $J \gets 2*facet.normal \times ray.amplitude(facet)$
    \State $currents.push(J)$
    \State $newRay.direction \gets ray.direction - 2*(ray.direction \cdot facet.normal)*facet.normal$
    \State $newRay.amplitude \gets \frac{0.5}{\left(facet.normal\cdot newRay.direction\right)} J \times newRay.direction$
    \State $rayBuffer.push( newRay )$
\EndIf
\EndWhile
\end{algorithmic}
\end{algorithm}


Note to self: Beware the negative signs negative signs!!

Equation in MOM regions:
\begin{equation}
	2\hat{n}\times\mathbf{H}_i(x)
	= \left( \mathbf{J}(x)
	- 2\hat{n}\times\int_{\Omega_m} \nabla_{x} G(x,x') \times \mathbf{J}(x') dS'\right)
	- \sum_n 2\hat{n}\times\int_{\Omega_n} \nabla_{x} G(x,x') \times \mathbf{J}(x') dS'.
\end{equation}


\end{document}
